
\section{Bend, or Break}

\subsection{Decoupling and the Law of Demeter}

\textbf{Introduction:} \\
Introduces the Law of Demeter as a guideline for reducing coupling between software components, advocating for loose coupling to enhance flexibility and maintainability.

\vspace{2mm}
\noindent\textbf{Key Concepts:}
\begin{itemize}
  \item \textit{Loose Coupling}: Discusses the benefits of minimizing dependencies between components, which allows for easier changes and better system robustness.
\end{itemize}

\vspace{2mm}
\noindent\textbf{Conclusion:} \\
Recommends adherence to the Law of Demeter to prevent complexity and rigidity in software designs, promoting designs that are easier to modify and extend.

\subsection{Metaprogramming}

\textbf{Introduction:} \\
Explores metaprogramming, the technique of writing programs that write or manipulate other programs, and its role in creating more adaptable and concise code.

\vspace{2mm}
\noindent\textbf{Key Concepts:}
\begin{itemize}
  \item \textit{Code Generation}: Emphasizes how metaprogramming can be used to generate code dynamically, leading to more scalable and reusable software.
\end{itemize}

\vspace{2mm}
\noindent\textbf{Conclusion:} \\
Advocates for the strategic use of metaprogramming to write flexible and DRY code, reducing the risk of errors and redundancy.

\subsection{Temporal Coupling}

\textbf{Introduction:} \\
Discusses temporal coupling in software systems, where the order of operations can lead to hidden dependencies and potential issues.

\vspace{2mm}
\noindent\textbf{Key Concepts:}
\begin{itemize}
  \item \textit{Order of Execution}: Addresses the importance of being aware of and minimizing temporal dependencies to make the code more understandable and maintainable.
\end{itemize}

\vspace{2mm}
\noindent\textbf{Conclusion:} \\
Encourages developers to recognize and manage temporal coupling to prevent insidious bugs and improve the clarity of the code.

\subsection{It's Just a View}

\textbf{Introduction:} \\
Examines the concept of views as a mechanism to separate the underlying data from the way it is presented to the user, enhancing the modularity of the code.

\vspace{2mm}
\noindent\textbf{Key Concepts:}
\begin{itemize}
  \item \textit{Presentation Abstraction}: Highlights the separation of concerns, demonstrating how views can abstract the presentation layer from data and logic.
\end{itemize}

\vspace{2mm}
\noindent\textbf{Conclusion:} \\
Promotes the use of views to isolate changes to the user interface from the core logic, facilitating easier updates and testing.

\subsection{Blackboards}

\textbf{Introduction:} \\
Covers the blackboard architectural pattern that enables a system to evolve its solution to a problem incrementally with contributions from different sources.

\vspace{2mm}
\noindent\textbf{Key Concepts:}
\begin{itemize}
  \item \textit{Collaborative Problem Solving}: Discusses how the blackboard pattern can facilitate collaboration among loosely coupled subsystems to solve complex problems.
\end{itemize}

\vspace{2mm}
\noindent\textbf{Conclusion:} \\
Endorses the blackboard pattern for complex problems where the solution is emergent and benefits from the contribution of diverse knowledge sources.
