\section{While You Are Coding}

\subsection{Programming by Coincidence}

\textbf{Introduction:} \\
Discusses the dangers of relying on accidental successes in coding, known as 'programming by coincidence,' which can lead to unreliable and unpredictable code.

\vspace{2mm}
\noindent\textbf{Key Concepts:}
\begin{itemize}
  \item \textit{Intentional Programming}: Advocates for understanding the reasoning behind code functionality rather than relying on guesswork or coincidences.
\end{itemize}

\vspace{2mm}
\noindent\textbf{Conclusion:} \\
Encourages programmers to rely on knowledge and planned actions rather than chance, to write more reliable and maintainable code.

\subsection{Algorithm Speed}

\textbf{Introduction:} \\
Covers the importance of algorithm efficiency and the impact of algorithm choice on the performance of software.

\vspace{2mm}
\noindent\textbf{Key Concepts:}
\begin{itemize}
  \item \textit{Performance Optimization}: Focuses on selecting and optimizing algorithms for speed to improve the overall efficiency of the software.
\end{itemize}

\vspace{2mm}
\noindent\textbf{Conclusion:} \\
Stresses the significance of understanding and analyzing algorithm speed within the context of the entire application's performance.

\subsection{Refactoring}

\textbf{Introduction:} \\
Introduces refactoring as a disciplined technique for restructuring an existing body of code, altering its internal structure without changing its external behavior.

\vspace{2mm}
\noindent\textbf{Key Concepts:}
\begin{itemize}
  \item \textit{Code Improvement}: Describes methods for incrementally improving the design of existing code, making it easier to understand and modify.
\end{itemize}

\vspace{2mm}
\noindent\textbf{Conclusion:} \\
Advocates for regular refactoring as a vital part of the software development process to keep code healthy and maintainable.

\subsection{Code That's Easy to Test}

\textbf{Introduction:} \\
Highlights the benefits of writing code that is easy to test and the role of tests in ensuring code quality and reliability.

\vspace{2mm}
\noindent\textbf{Key Concepts:}
\begin{itemize}
  \item \textit{Testability}: Discusses strategies for writing code that is easy to test, such as decoupling and using test doubles.
\end{itemize}

\vspace{2mm}
\noindent\textbf{Conclusion:} \\
Emphasizes the importance of testability in code to catch errors early and to ensure that future changes do not introduce new bugs.

\subsection{Evil Wizards}

\textbf{Introduction:} \\
Addresses the temptation to use 'wizardry' or complex solutions that may seem clever but can often lead to code that is hard to understand and maintain.

\vspace{2mm}
\noindent\textbf{Key Concepts:}
\begin{itemize}
  \item \textit{Simplicity over Complexity}: Urges the avoidance of overly clever tricks in favor of clear, straightforward, and maintainable code.
\end{itemize}

\vspace{2mm}
\noindent\textbf{Conclusion:} \\
Concludes with the recommendation to resist the allure of 'coding wizardry' and to favor simplicity and clarity in programming practices.
